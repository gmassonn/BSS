\section{Introduction}
\textcolor{red}{This is a copy of the introduction of the draft sent to the quanticol project. TO BE MODIFIED.}

\subsection{Motivation and description of the system}
In recent years, bicycle sharing systems (BSS) have proven to be very
successful in several major cities and are now spreading all accross
the world.  Nowadays, there exist more than 700 such systems that
operate on five continents The goal for the cities are multiples: From
a greener image due to more eco-friendly means of transportation to
the reduction of traffic congestion, noise and air pollution, they
provide an alternative to private motorised vehicule, especially for
short-distance trips.  From the user's perspective, they offer an
affordable and efficient mean of transportation with several benefits
over the use of a personal bicycle regarding maintenance, theft or
storage issues.

A bike-sharing network is composed of stations where a limited number
of bikes can be parked.  The dynamics are simple: Each user can arrive
at a station to pick up a bike and can travel with it to another
station.  If the station at the origin is empty, the user can either
leave the system or try to find a bike in another station of the
network.  If the station at the destination is full, the user must
return the bike to another station with at least one parking spot
available.

Several issues arise in practice, the most important being the lack of
resource: if no bikes are available at the origin or no parking spot
are available at the destination, the mechanics described above forces
users to adapt to shortage or blocking situations without notice prior
to their arrival at the stations, leading to trip delays or
cancellations.  Hence in practice, BSS companies often use trucks to
dynamically reallocate bikes in the system and avoid problematic
situation, in order to reach the quality and efficiency of service
targeted by the cities.  Such policies turned out to be costly and
unsuitable for specific issues and many challenges remain unanswered,
particularly on how to design and manage these systems and increase
their usage rate by giving the users more incentives to embrace them.

Our goal is this work is to use a mathematical model to predict the
possibilities for users to perform a trip. By using the measured
occupancy of the stations and past user behavior, we want to answer
questions like:
\begin{itemize}
\item \emph{``what is the probability that a trip from A to B is
    possible at 6pm a regular Monday''}? 
\item \emph{``how much detour should I plan if I want to go from A to
    B in 30min?''}
\item \emph{``If I want to arrive at the train station at 9am, when
    should I leave home and where should I pick or park my bike?''}
\end{itemize}

\subsection{Literature review}

The research related to such models is still in its early stages, but
publication over the subject have increased recently.  Some of the
work focuses on a social analysis of BSS and how such systems
influence global transportation system in several cities across the
world.  Data analysis is another important aspect of the research,
since it allows us analyse customers' usage of the system and reaction
to problematic situation, as well as trying to forecast their demands.
Finally, the study of BSS have lead to new mathematical models and
optimization techniques, either by by focusing on the design of the
system itself or on bike allocation and movement through the system.
Table~\ref{table:LitRev} summarizes some of the literature we are
aware of.


\begin{table}
\centering
\begin{tabular}{||c|c|c||}
\hline
\hline
\multirow{3}{*}{Social studies of BSS} & History & \cite{DeMaio2009}\\ 
\cline{2-3}
&
 Users' behavior & \begin{minipage}{0.4\textwidth}\cite{Froehlich2008}, \cite{Shaheen2010}\end{minipage}\\
\cline{2-3}
&
 Case studies & \begin{minipage}{0.4\textwidth}\cite{Shaheen2007}, \cite{Shaheen2011}, \cite{Correia2014}\end{minipage}\\
\hline
\hline
\multicolumn{2}{||c|}{Data analysis} &  \begin{minipage}{0.4\textwidth}\cite{Froehlich2008}, \cite{Borgnat2011}, \cite{Nair2013}, \cite{Frade2014}\end{minipage}\\
\hline
\hline
\multicolumn{2}{||c|}{Mathematical modelization} & \begin{minipage}{0.4\textwidth}\cite{Shu2005}, \cite{Fricker2012}, \cite{Fricker2014}\end{minipage}\\ 
\hline
\hline
\multirow{4}{*}{Optimization techniques} & System design &\begin{minipage}{0.4\textwidth}\cite{Campbell1992}, \cite{Katzev2003}, \cite{DeMaio2004}, \cite{Borndorfer2007}, \cite{Correia2012}\end{minipage}\\ 
\cline{2-3}
&
 Fleet sizing and assignment & \begin{minipage}{0.4\textwidth}\cite{George2011}, \cite{Raviv2013a}, \cite{Shu2013}\end{minipage}\\
\cline{2-3}
&
 Bikes repositioning & \begin{minipage}{0.4\textwidth}\cite{Kek2009}, \cite{Benchimol2011}, \cite{Chemla2012}, \cite{Contardo2012}, \cite{Raviv2013b}, \cite{Schuijbroeck2013}, \cite{Angeloudis2014}\end{minipage}\\
\cline{2-3}
&
 Pricing techniques 
& \begin{minipage}{0.4\textwidth}\cite{Waserhole2012}, \cite{Waserhole2013}\end{minipage}\\
\hline
\hline
\end{tabular}
\caption{Overview of some literature over VSS}
\label{table:LitRev}
\end{table}

%%% Local Variables:
%%% mode: latex
%%% TeX-master: "Draft_BSS_QUANTICOL"
%%% End:
