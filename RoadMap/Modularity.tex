\section{Modularity between different models}

\subsection{Advanced reservation}
\begin{enumerate}
\item {\em Without reservation}: The users pick up bikes independently and at their own convenience, regardless of any prior ``aproval'' from the system.
\item {\em With full reservation}: The users submit a request to the system, which in turn delivers -- or not -- an authorization for the trip requested, by blocking resources (vehicules at the origin and/or parking spots at the destination).
Two types of reservations: 
\begin{itemize}
\item[$\rightarrow$] At the station at the origin of the trip, blocking a parking spot at the destination.
\item[$\rightarrow$] Beforehand, booking a trip blocks a vehicule at the station of origin and a free parking spot at the destination.
\end{itemize}
\item {\em Overbooking}: Still unclear. Allow more reservation than the resources available. Can we obtain a better overall control of the system with this approach?
\end{enumerate}

\subsection{Temporality considerations}
\begin{enumerate}
\item {\em Stationary systems}: All parameters are time-independent. Easier?
\item {\em Discrete time steps}: Finite or infinite time-horizon?
\item {\em Continuous time}
\end{enumerate}

\subsection{Spatial/geographical considerations}
\begin{enumerate}
\item {\em Homogeneous parameters (``perfect cities'')}: 
\end{enumerate}
